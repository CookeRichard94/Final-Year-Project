%!TEX root = project.tex

\chapter*{About this project}
\paragraph{Abstract}
A brief description of what the project is, in about two-hundred and fifty words.

\paragraph{Authors}
This project was designed and built by Mark Gill and Niema Attarian.


\chapter{Introduction}
3-5 pages

\chapter{Context}
\begin{itemize}
\item Provide a context for your project.
\item Set out the objectives of the project
\item Briefly list each chapter / section and provide a 1-2 line description of what each section contains.
\item List the resource URL (GitHub address) for the project and provide a brief list of the main elements at the URL.
\end{itemize}

\section{Filler}

\subsection{More filler}

\section{Filler}


\chapter{Methodology}

\section{Incremental and iterative approach to development}
Initially, when this project began with three members, we discussed how we would break down our idea both between us and with our project supervisor. Our first idea consisted of a commercial Angular application, similar to that of a business to consumer application like Amazon or eBay. Our idea consisted of a front-end, back-end, a neural network which took in a picture that identified a particular product like shoes, a t-shirt, etc and the cloud hosting. Along with advice from our supervisor,  we agreed on sharing the responsibility of the project as equally as possible with Niema designing the front-end of the application, Mark coding the back-end of the application, Richard in charge of the neural network and the cloud hosting being a final addition towards the end of this project.

We met two-to-three times as week as a group to discuss our approach and any issues we had. We also routinely had weekly meetings, on a Monday morning, with our supervisor Gerard to discuss ideas, frameworks and various methods on how to approach this project. We would detail our progression in these meetings with Gerard and also mention any outstanding issues.
We kept a very open mindset in developing our project. This was clear in our idea choice. It was extremely versatile which helped with any suggested changes or approaches which differed from our base idea. 

However, as the weeks continued and further discussion with our supervisor, we mutually agreed that our idea and workload was not sufficient for a three person group. Therefore, we disbanded into a two-person group of Niema and Mark. Following the advice of our supervisor, we discussed how we would restructure our project and re-assign the workload evenly. The scope of the project barely changed with the only difference being the removal of the neural network aspect. We did, however, decide to change the language we planned on developing the application from Angular to Django.

\section{Selection criteria for Technologies, Frameworks and Platforms}

\subsection{Technologies}
\paragraph{TypeScript}
Our initial approach included using TypeScript[] to build our commercial application. We believed that this language was best suited for our project as it is essentially super-set of JavaScript, a language we are familiar with and have used consistently over our four years of software development. It has the ability to enable easier development on a large scale which was essential to our versatile and expandable idea\cite{bierman_abadi_torgersen_2014}.

\paragraph{Python}


\paragraph{HTML}
HTML was agreed upon for the front-end of the project. This choice was made mainly for it's simplicity. It is easily implemented and designed.

\paragraph{CSS}
CSS was agreed upon for the design of the front-end of the project. This choice was made mainly for it's simplicity. 

\paragraph{JavaScript}
This choice was made mainly for it's simplicity.

\subsection{Frameworks}
\paragraph{Angular}
With our first ideas development including TypeScript, we found that Angular[] to be the best framework for our large-scale, single page application. Angular was a familiar framework to us, one that we have used for the course of two years.  

\paragraph{Django}


\paragraph{Flask}


\subsection{Platforms}
\paragraph{GitHub}
As a team based final-year project, we unanimously agreed that GitHub\cite{github} would be best suited to manage our project. This decision was without hesitation as GitHub was used widely in our four years of studying Software Development. GitHub is widely used amongst every profession of software development and it is evident that the platform has helped improve the way in which software professionals work and collaborate\cite{zagalsky2015emergence}.

\section{What about validation and testing?}
Testing the application was very important to us. We wanted to ensure that it functioned smoothly and accurately. We were torn on the various ways to test our application; from using Cucumber testing, JUnit tests or the Selenium framework.

We concluded that Selenium would be the preferred testing framework to test our application.


\chapter{Technology Review}
About seven to ten pages.
\begin{itemize}
\item Describe each of the technologies you used at a conceptual level. Standards, Database Model (e.g. MongoDB, CouchDB), XMl, WSDL, JSON, JAXP.
\item Use references (IEEE format, e.g. [1]), Books, Papers, URLs (timestamp) – sources should be authoritative. 
\end{itemize}

\section{XML}
Here's some nicely formatted XML:
\begin{minted}{xml}
<this>
  <looks lookswhat="good">
    Good
  </looks>
</this>
\end{minted}

\chapter{System Design}
As many pages as needed.
\begin{itemize}
\item Architecture, UML etc. An overview of the different components of the system. Diagrams etc… Screen shots etc.
\end{itemize}

\begin{table}[h]
  \centering
  \begin{tabular}{x{2cm}p{3cm}}
    \toprule \\
    Column 1 & Column 2 \\
    \midrule \\
    Rows 2.1 & Row 2.2 \\
    \bottomrule
  \end{tabular}
  \caption{A table.}
  \label{table:mytable}
\end{table}

\chapter{System Evaluation}
As many pages as needed.
\begin{itemize}
\item Prove that your software is robust. How? Testing etc. 
\item Use performance benchmarks (space and time) if algorithmic.
\item Measure the outcomes / outputs of your system / software against the objectives from the Introduction.
\item Highlight any limitations or opportuni-ties in your approach or technologies used.
\end{itemize}

\chapter{Conclusion}
About three pages.

\begin{itemize}
\item Briefly summarise your context and ob-jectives (a few lines).
\item Highlight your findings from the evalua-tion section / chapter and any opportuni-ties identified.
\end{itemize}

https://help.github.com/en/desktop

